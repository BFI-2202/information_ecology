\documentclass{article}

\usepackage[T2A]{fontenc}
\usepackage[utf8]{inputenc}
\usepackage[russian]{babel}

\usepackage{tabularx}
\usepackage{amsmath}
\usepackage{pgfplots}
\usepackage{geometry}
\usepackage{multicol}
\geometry{
    left=1cm,right=1cm,top=2cm,bottom=2cm
}
\newcommand*\diff{\mathop{}\!\mathrm{d}}

\newtheorem{definition}{Определение}
\newtheorem{theorem}{Теорема}

\DeclareMathOperator{\sign}{sign}

\usepackage{hyperref}
\hypersetup{
    colorlinks, citecolor=black, filecolor=black, linkcolor=black, urlcolor=black
}

\title{Информационная экология}
\author{Лисид Лаконский}
\date{Февраль 2023}

\begin{document}
\raggedright

\maketitle

\tableofcontents
\pagebreak

\section{Практическое занятие — 05.02.2024}

\subsection{Введение}

\textbf{Преподаватель} — Жукова Жанна Сергеевна, старший преподаватель кафедры «Экология, безопасность жизнедеятельности и электропитание», контактный телефон: 8-977-929-56-21.

В семестре будет 12 лекций (1 б. — за посещение лекций + 3/2 б. — за тестики по лекции, проводящиеся в конце лекции по ее материалам; конспекты вести можно), 12 лабораторных работ (1 б. — за посещение + 3 б. — за написание лабораторной + 3 б. — интерактив, разговорные навыки).

Максимально возможное количество баллов: 162. Итоговый тест — 30 баллов. Зачёт — 160 баллов. Баллы можно получать с помощью всяких докладов и участия в олимпиадах и конференциях.

\end{document}