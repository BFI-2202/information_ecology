\documentclass{article}

\usepackage[T2A]{fontenc}
\usepackage[utf8]{inputenc}
\usepackage[russian]{babel}

\usepackage{tabularx}
\usepackage{amsmath}
\usepackage{pgfplots}
\usepackage{geometry}
\usepackage{multicol}
% \geometry{
%     left=1cm,right=1cm,top=2cm,bottom=2cm
% }
\newcommand*\diff{\mathop{}\!\mathrm{d}}

\newtheorem{definition}{Определение}
\newtheorem{theorem}{Теорема}

\DeclareMathOperator{\sign}{sign}

\usepackage{hyperref}
\hypersetup{
    colorlinks, citecolor=black, filecolor=black, linkcolor=black, urlcolor=black
}

\title{Информационная экология}
\author{Лисид Лаконский}
\date{Февраль 2023}

\begin{document}
\raggedright

\maketitle

\tableofcontents
\pagebreak

\section{Лекция — 06.04.2024}

\subsection{О предмете экологии}

\textbf{Экология} — наука о живых организмах, их взаимодействии друг с другом и окружающей природной средой. Ввёл это понятие \textbf{Геккель} в \textbf{1866 г.}

\textbf{Биосфера} — особая оболочка Земли, где обитают живые организмы. Учение \textbf{Вернадского}. Понятие ввел \textbf{Зюсс}. Включает в себя:

\begin{enumerate}
    \item \textbf{Гидросфера} — все воды Земли. Является полностью заселённой.
    \item \textbf{Атмосфера} — воздух; до 10-300 км, но самолеты, бывает, летают выше. И МКС выше расположено. не является полностью населенной.
    \item \textbf{Литосфера} — земля; до 6 км под землёй, 0–4 над землей. Не является полностью населённой.
\end{enumerate}

\textbf{Вид} — совокупность организмов, обладающих общим генофондом. Виды делятся на \textbf{популяции} — совокупность их представителей, проживающих обособленно. Популяции чаще всего скрещиваются друг с другом.

\textbf{Биоценоз} — совокупность живых организмов на какой-то сходной территории; проживающих в каких-то общих условиях. \textbf{Биотоп} — неживая его составляющая.

\textbf{Биогеоценоз} — территория, заселенная различными видами.

\textbf{Экосистемой} называется совокупность живых организмов на определенной территории с учётом климатических взаимодействий.

\subsection{Трофические взаимодействия}

\textbf{Растения} занимаются \textbf{фотосинтезом} (CO2, H20, микроэлементы и солнышко), являются производителем \textbf{органики} (то есть, являются \textbf{продуцентами}). Являются \textbf{автотрофами} — автономно питаются. В противоположность, \textbf{гетеротрофы} используют уже запасенное органическое вещество.

\textbf{Продуценты} (растения) получают солнечную энергию, из окружающей среды к ним приходит CO2, H2O, другие элементы.

\textbf{Консументы} — потребители первичной продукции, производящие вторичную продукцию. \textbf{Фитофаги} — травоядные, консументы I-го р. \textbf{Зоофаги} — консументы II-го р.

\textbf{Редуценты} — простейшие грибы и бактерии; организмы, разрушающие отмершие останки живых существ, превращая их в неорганические и простейшие органические соединения. В экологии редуцентами также считаются \textbf{детритофаги} — животные, питающиеся разлагающейся органикой; черви, жук навозник.
 
\textbf{Зоофаги}: \textbf{хищные} — плацентарные млекопетающие, морские млекопитающие (кошечки, совы, морские котики, косатки), \textbf{хищническое поведение} — те, кто ловят и умерщвляют других животных, \textbf{сверххищники} — доминирующий вид в какой-то экосистеме; например, в Арктике — медведь; в Африке — большие кошки.

\textbf{Техносферой} называется искуственная оболочка Земли. Сверххищники являются \textbf{доминирующим видом} на определенной территории. Человек — единственный вид, \textbf{доминирующий} на всей планете. 

\end{document}